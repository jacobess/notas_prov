\chapter{Conceptos importantes}

%Agregar una introducción mas ``motivadora''
Computadora : M\'aquina electronica de c\'alculo, compuesta por circuitos l\'ogicos que generan conexiones.
\\ \\
Componentes de circuitos lógicos 

\begin{itemize}
\item Amplificador operacional:
\item Biestable:
\item PDL:
\item Diac:
\item Diodo:
\item FGPA: 
\item Memoria:
\item Microprocesador:
\item Pila:
\item Tiristor:
\item Puerta l\'ogica:
\item Transistor:
\item Triac: 
\end{itemize}

%----Faltan imagenes de los elementos----%

%----Tabla de almacenamiento de datos-----%

Representaci\'on de punto flotante \\
%----Ilustración de celdas-----%
$a \times 10^b$\\
$1 \textgreater |a| \geq 0.1$\\
exceptuando cuando $a=0.0b$ \\

\begin{tabular}{| c | c |}
\hline
Tipo de datos & espacio de almacenamiento\\
\hline 
float& 4 bytes\\
double & 8 bytes\\
long double & 16 bytes\\
\hline
\end{tabular}
\\ \\
Error de corte:
\\ \\
Epsil\'on de la m\'aquina (EPS): El EPS es el n\'umero m\'as pequeño tal que $(1+EPS)\textgreater1$ para la m\'aquina que realiza la suma.
El siguiente c\'odigo permite conocer el epsil\'on de tu computadora.
%----Código para el epsilón----%
\\ \\
Error de redondeo: P\'erdida de cifras decimales a medida que se aumenta el exponente.
%---Ilustración de recta numérica---%
\\ \\
Error de truncamiento \\
Teorema de Taylor: Si $f(x)$ es una funci\'on suave en un intervalo abierto $(a,b)$ que contiene a $c$, para un n\'umero $c+h$ contenido en $(a,b)$ \\
$f(c+h)=f(c)+f'(c)h+f''(c)\frac{h^3}{3!}+f'''(c)\frac{h^3}{3!}+...+f^n(c)\frac{h^n}{n!}$ \\
El hecho de perder cifras debido a limitar el resultado a ciertas decimale es a lo que llamamos error de truncamiento.
\\ \\
Complejidad algor\'itmica y costo computacional:
%---Diagramas---%
\\ \\ 
Tiempo de tendencia a funciones\\
\begin{tabular}{| c | c |}
\hline
$log(n)$ & \\
$n$ & Tiempos lineales \\ 
$nlog(n)$ & \\
\hline
$n^2$ & \\
$n^3$ & Tiempos polin\'omicos \\ 
$n^4$ & \\
\hline
$2^n$ & NP-Duro \\
$n!$ & NP-Cmpleto\\
\hline
\end{tabular}
%-Ilustracioes de graficas-%
\\ \\
Convergencia\\
Un ciclo de c\'alculo se traduce a una iteraci\'on 
%--Ilustración de iteración--%
\\
Sea $x_k$ una sucesión de valores. Si existe un n\'umero $x^*$ tal que\\
$\lim\limits_{k\to\infty}x_k=x^*$ \\
La sucesi\'on converge a $x^*$, si eso no ocurre, entonces la suseci\'on diverge.
\\ \\ 
Velocidad de convergencia\\ 
${x_k}$ converge s $x^*$ \\
\begin{enumerate}
\item Si existe un $k\geq 1$ a partir del cual se observa que $|x_{k+1}-x^*|\leq C|x_k-x^*|$ donde $C$ es constante entre $(0,1)$ se tiene velocidad de convergencia lineal.
\item Igual que el anterior pero $|x_{k+1}-x^*|\leq c_k|x_k-x^*|$ con $c_k\exists(0,1)$ y $C_k\to0$ cuando $k\to\infty$, se tiene  velocidad de convergencia lineal.
\item A partir de la iteraci\'on $k$ se observa que\\
$|x_{k+1}-x^*|\leq C{|x_k-x^*}^P$ donde $C$y $P$ son constantes $C\exists(0,1)$ y $P\geq2$, se tiene convergencia de orden $P$
\end{enumerate}
